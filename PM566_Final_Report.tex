% Options for packages loaded elsewhere
\PassOptionsToPackage{unicode}{hyperref}
\PassOptionsToPackage{hyphens}{url}
\PassOptionsToPackage{dvipsnames,svgnames,x11names}{xcolor}
%
\documentclass[
  letterpaper,
  DIV=11,
  numbers=noendperiod]{scrartcl}

\usepackage{amsmath,amssymb}
\usepackage{iftex}
\ifPDFTeX
  \usepackage[T1]{fontenc}
  \usepackage[utf8]{inputenc}
  \usepackage{textcomp} % provide euro and other symbols
\else % if luatex or xetex
  \usepackage{unicode-math}
  \defaultfontfeatures{Scale=MatchLowercase}
  \defaultfontfeatures[\rmfamily]{Ligatures=TeX,Scale=1}
\fi
\usepackage{lmodern}
\ifPDFTeX\else  
    % xetex/luatex font selection
\fi
% Use upquote if available, for straight quotes in verbatim environments
\IfFileExists{upquote.sty}{\usepackage{upquote}}{}
\IfFileExists{microtype.sty}{% use microtype if available
  \usepackage[]{microtype}
  \UseMicrotypeSet[protrusion]{basicmath} % disable protrusion for tt fonts
}{}
\makeatletter
\@ifundefined{KOMAClassName}{% if non-KOMA class
  \IfFileExists{parskip.sty}{%
    \usepackage{parskip}
  }{% else
    \setlength{\parindent}{0pt}
    \setlength{\parskip}{6pt plus 2pt minus 1pt}}
}{% if KOMA class
  \KOMAoptions{parskip=half}}
\makeatother
\usepackage{xcolor}
\setlength{\emergencystretch}{3em} % prevent overfull lines
\setcounter{secnumdepth}{-\maxdimen} % remove section numbering
% Make \paragraph and \subparagraph free-standing
\ifx\paragraph\undefined\else
  \let\oldparagraph\paragraph
  \renewcommand{\paragraph}[1]{\oldparagraph{#1}\mbox{}}
\fi
\ifx\subparagraph\undefined\else
  \let\oldsubparagraph\subparagraph
  \renewcommand{\subparagraph}[1]{\oldsubparagraph{#1}\mbox{}}
\fi


\providecommand{\tightlist}{%
  \setlength{\itemsep}{0pt}\setlength{\parskip}{0pt}}\usepackage{longtable,booktabs,array}
\usepackage{calc} % for calculating minipage widths
% Correct order of tables after \paragraph or \subparagraph
\usepackage{etoolbox}
\makeatletter
\patchcmd\longtable{\par}{\if@noskipsec\mbox{}\fi\par}{}{}
\makeatother
% Allow footnotes in longtable head/foot
\IfFileExists{footnotehyper.sty}{\usepackage{footnotehyper}}{\usepackage{footnote}}
\makesavenoteenv{longtable}
\usepackage{graphicx}
\makeatletter
\def\maxwidth{\ifdim\Gin@nat@width>\linewidth\linewidth\else\Gin@nat@width\fi}
\def\maxheight{\ifdim\Gin@nat@height>\textheight\textheight\else\Gin@nat@height\fi}
\makeatother
% Scale images if necessary, so that they will not overflow the page
% margins by default, and it is still possible to overwrite the defaults
% using explicit options in \includegraphics[width, height, ...]{}
\setkeys{Gin}{width=\maxwidth,height=\maxheight,keepaspectratio}
% Set default figure placement to htbp
\makeatletter
\def\fps@figure{htbp}
\makeatother

\KOMAoption{captions}{tableheading}
\makeatletter
\makeatother
\makeatletter
\makeatother
\makeatletter
\@ifpackageloaded{caption}{}{\usepackage{caption}}
\AtBeginDocument{%
\ifdefined\contentsname
  \renewcommand*\contentsname{Table of contents}
\else
  \newcommand\contentsname{Table of contents}
\fi
\ifdefined\listfigurename
  \renewcommand*\listfigurename{List of Figures}
\else
  \newcommand\listfigurename{List of Figures}
\fi
\ifdefined\listtablename
  \renewcommand*\listtablename{List of Tables}
\else
  \newcommand\listtablename{List of Tables}
\fi
\ifdefined\figurename
  \renewcommand*\figurename{Figure}
\else
  \newcommand\figurename{Figure}
\fi
\ifdefined\tablename
  \renewcommand*\tablename{Table}
\else
  \newcommand\tablename{Table}
\fi
}
\@ifpackageloaded{float}{}{\usepackage{float}}
\floatstyle{ruled}
\@ifundefined{c@chapter}{\newfloat{codelisting}{h}{lop}}{\newfloat{codelisting}{h}{lop}[chapter]}
\floatname{codelisting}{Listing}
\newcommand*\listoflistings{\listof{codelisting}{List of Listings}}
\makeatother
\makeatletter
\@ifpackageloaded{caption}{}{\usepackage{caption}}
\@ifpackageloaded{subcaption}{}{\usepackage{subcaption}}
\makeatother
\makeatletter
\@ifpackageloaded{tcolorbox}{}{\usepackage[skins,breakable]{tcolorbox}}
\makeatother
\makeatletter
\@ifundefined{shadecolor}{\definecolor{shadecolor}{rgb}{.97, .97, .97}}
\makeatother
\makeatletter
\makeatother
\makeatletter
\makeatother
\ifLuaTeX
  \usepackage{selnolig}  % disable illegal ligatures
\fi
\IfFileExists{bookmark.sty}{\usepackage{bookmark}}{\usepackage{hyperref}}
\IfFileExists{xurl.sty}{\usepackage{xurl}}{} % add URL line breaks if available
\urlstyle{same} % disable monospaced font for URLs
\hypersetup{
  pdftitle={PM566 Final},
  pdfauthor={Sylwia Lipior},
  colorlinks=true,
  linkcolor={blue},
  filecolor={Maroon},
  citecolor={Blue},
  urlcolor={Blue},
  pdfcreator={LaTeX via pandoc}}

\title{PM566 Final}
\author{Sylwia Lipior}
\date{}

\begin{document}
\maketitle
\ifdefined\Shaded\renewenvironment{Shaded}{\begin{tcolorbox}[breakable, borderline west={3pt}{0pt}{shadecolor}, frame hidden, interior hidden, boxrule=0pt, sharp corners, enhanced]}{\end{tcolorbox}}\fi

\hypertarget{introduction}{%
\subsection{Introduction}\label{introduction}}

Catapult devices are GPS trackers worn by athletes. Wearable-based
tracking technologies are used throughout sport to support performance
monitoring. In addition to GPS capability, these devices contain
inertial sensors comprising of an accelerometer (to measure acceleration
forces), a gyroscope (to measure rotation), and a magnetometer (to
measure body orientation). Inertial sensors collect data in three axes,
or directions, allowing sensitive `maps' of athlete movements and
actions to be created. For Catapult's website, they claim: ``The
combination of the wearable tracking device and the inertial sensors
creates a powerful athlete monitoring tool that ensures that key
performance decisions are always supported with objective data.'' The
sports performance department at LA Galaxy uses Catapult data to make
decisions about performance readiness, rehabilitation, and training
prescription.

This data is Catapult data collected over the course of the U17
2022-2023 season. I will specifically be look at U17 game data for that
season. As a student in the USC Sports Science program and an intern at
the LA Galaxy Sports Performance Department, I've had the opportunity to
assist with collecting this data since January 2023. This data is
typically visualized using either Catapult's Cloud where they offer many
widgets to visualize data, or an internal athlete management system. LA
Galaxy has been developing an athlete management system using Microsoft
Azure. They export the data from Catapult and import it to Azure and
have customized many different dashboards to visualize data. For this
project, I decided to export CSVs directly from Catapult and try to
wrangle the data myself.

Catapult data is collected at every training session and game. The
players wear the devices on vests produced by Catapults and the GPS
units are stored in a secure pouch on the back of the vest. During
training or games, a member of the sport performance department will
have an iPad which has the Vector app created by Catapult. The Vector
app allows the user to input information about the training session or
game, and it produces a live view of the Catapult data per player. The
user can start and stop ``Periods'' based on training drills and which
players are participating. After training, all the units are collected,
put into a dock, and uploaded to a computer. This data is then available
in the Cloud and can be exported to the athlete management system for
further visualization.

When thinking about this data, my research question became: does fatigue
affect player's physical performance in soccer matches? More
specifically, are players less physically productive when they are
tired? To answer this question, I looked at the data at a few levels. To
start off, I look at the difference in player's maximum velocities in
the first half of games vs the second half. Then, I look at a string of
five games in seven games that the team played in difficult conditions
in the MLS Next Tournament, which was played in June of this year.

\hypertarget{methods}{%
\subsection{Methods}\label{methods}}

\hypertarget{preparing-the-data-frame}{%
\subsubsection{Preparing the data
frame}\label{preparing-the-data-frame}}

When you export bulk CSVs from Catapult, you get observations for every
player involved in the session for 1699 variables. A lot of that data is
a little redundant, but I wrote a function to subset the data with only
around 34 variables of interest to make it more manageable. The CSVs
don't have the activity name easily accessible, so I wrote a function to
extract the names from the names of the CSV files. I then wrote a for
loop to read in all the data (\textasciitilde57 CSVs which corresponds
to data from 57 games), making a new variable for the date of the
session, and a new variable for the activity name. Finally, I
de-identified the data since the data contains player names.

\hypertarget{initial-visualization-of-the-data}{%
\subsection{Initial visualization of the
data}\label{initial-visualization-of-the-data}}

Next, I wanted to make sure the data looks how I would expect it to.
Since the session names are inputted by staff, there is some room for
error, and I wanted to make sure I have only game data here. To
accomplish this, I decided to plot maximum velocity for each activity by
month.

From looking at the initial box plots, I figured out that there is some
data in the set that doesn't belong. Specifically, ``U17 GD vs RSL''
seems to have an issue with the GPS data since the maximum velocities
are so low, so I decided to remove that data. In addition, the maximum
velocities for ``U17 GD Pre Season Day 2'' are much lower than expected.
Once I looked at the period names, I realized that this is data from a
training session that was mislabeled as a game, so I removed it from the
data set. I included the code to produce these box plots, but decided
not to render here, because I did not use them for further analysis.

\hypertarget{player-load-scatter-plots}{%
\subsection{Player Load Scatter Plots}\label{player-load-scatter-plots}}

For some more initial visualization, I decided to look at the
relationship between ``Player Load'', which is defined by Catapult as
``the sum of the accelerations across all axes of the internal tri-axial
accelerometer during movement'', and a few other physical metrics.
Specifically, I looked at scatterplot of Player Load vs Total Distance
Covered, Total Number of Sprints, Explosive Efforts, and Total High
Intensity Bouts.

After looking at the plots, as expected there is a strong positive
correlation between Player Load and all the other physical metrics. In
other words, as a player covers more distance, or performs more sprints,
explosive efforts, or high intensity bouts, their player load is
expected to be higher.

\hypertarget{maximum-velocity-analysis}{%
\subsection{Maximum Velocity Analysis}\label{maximum-velocity-analysis}}

I thought it would be interesting to analyze maximum velocity from the
dataset. Unfortunately, the data is a little difficult, because there
are various Period Names that could signify either First Half or Second
Half game data. Therefore, I used mutate to add a variable called
``Period.Name.Halves'' to denote which observations are from the first
half vs the second half. I also removed Goal Keeper data, since their
data looks very different from field players due to the nature of their
position. I added a threshold of 10mph for speed, to make sure that I am
actually using game data, and not some other potentially mislabeled
data. I then visualized the maximum velocity data in a few different
ways.

The ``Comparison of Max Speed in First Half vs Second Half'' box plot
showed that there doesn't seem to be a big difference in the average
speed of players in the first half and second half. Something I could
consider is removing players that did not play a full game from the
data, since players are often subbed on in the second half. These
substitutes could be running faster since they aren't fatigued yet, so
they could be increasing the average. There seems to be a larger range
of speed in the second half, which makes sense since players are getting
fatigued.

Next, I was interested in looking at the individual player level. I took
the frequency of players that appeared in the top 3 max speed values per
activity. ``Player\_14'' is clearly the fastest player, since he appears
in the top 3 the most times out of anyone. I made a stacked box plot
looking at the five players that appeared in the top 3 for maximum
velocity the most number of times. Four of the players appear in the top
3 the most times for both the first half and second half, but player 3
appeared in the top 3 the third most times for first half and player 12
appeared in the top 3 the third most times for the second half. That
tells me that player 12 might be a second half substitute often while
player 3 gets subbed off in the second half.

I then made a spaghetti plot looking at the top recorded velocity for
the first half vs second half for each player to see if there is a
difference in performance between halves. Interestingly, the trend seems
to be that players achieve higher maximum velocities in the second half
of games. That makes sense, because player's might be experience fatigue
and might make errors where they have to achieve very high speeds to
deal with counter attacks. I made a similar spaghetti plot, except with
average velocity across all activities. Now, the trend seems to be the
opposite.

\hypertarget{extract-games-from-mls-next-tournament-june-2023}{%
\subsubsection{Extract Games from MLS Next Tournament (June
2023)}\label{extract-games-from-mls-next-tournament-june-2023}}

In June of 2023, the LA Galaxy U17 team won the MLS Next Tournament. In
order to hoist the trophy, they played 5 matches in 7 days in very
difficult and humid conditions in Dallas, Texas. To perform an analysis
on this data, I had to subset the master data frame to extract the data
for these matches. Then, I calculated the match totals for the following
physical metrics: total distance, total high intensity bouts, total
player load, total explosive efforts, total sprints, total high speed
distance (12-14 mph), total very high speed distance (14-17 mph), and
total sprinting distance (17-19 mph). I created a table summarizing this
data, and made multiple bar charts.

When looking at total distance covered, there doesn't seem to be a
decrease as the matches progress. Interestingly, the first match had the
most sprints, and almost the highest high speed running distance. This
is expected since this was the first match of the tournament. There was
a noticeable dip in the second match, but this was played the day after
the first match. There seemed to be some affects of fatigue in this
match. The high speed running distance was the lowest of the tournament.
The players also completed the fewest number of sprints in this match.
There is also a dip in the number of sprints in the last match, which
seems to be an affect of accumulated fatigue. Overall, all of the
physical metrics looked at here are have comparable values. When looking
at the distribution of speed ranges for each match, there was a
noticeable decrease in high speed running in the second match (the one
with the least rest) and the last match, again probably due to
accumulated fatigue. The distribution of sprinting speeds seems to be
fairly similar between matches. As expected, players cover the most
distance at high speed, and a small amount of distance at ``very'' high
speeds and sprinting speed.

\hypertarget{conclusion}{%
\subsection{Conclusion}\label{conclusion}}

From my analysis, I was able to get a better understanding of how
fatigue affects physical metrics measured by Catapult devices. In my
analysis of maximum velocity on the player level, I found that fatigue
within a single match does not seem to affect whether players will hit a
high maximum velocity. The average maximum velocity for the team is
slightly higher in the first half and in the second half, which is
expected. Intra-player differences in maximum velocity are very low
between the first half and the second half. Furthermore, in my analysis
of the MLS Next Tournament, I found that there was not a large impact of
accumulated fatigue. The distribution of different sprinting speed
distances was similar between the games (i.e.~player's did not seem to
be sprinting less as the tournament progressed). That's must mean that
players are good at recovering and have good fitness. The matches that
seemed the most affected by fatigue were the second match and the last
match. As discussed above, that is expected because the second match was
played the day after the first match. All of the other matches had at
least one day's rest in between. The last match presumably had lower
values due to accumulated fatigue. In conclusion, fatigue definitely has
an effect on player's physical performance, but this needs to be
analyzed further, and it varies on a case-by-case basis. High performing
athletes seem to be very good at recovering quickly and minimizing the
effects of fatigue.



\end{document}
